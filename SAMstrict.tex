\documentclass[10pt]{article}
\usepackage[margin=1in]{geometry}
\usepackage{longtable}
\usepackage[pdfborder={0 0 0},hyperfootnotes=false]{hyperref}
\usepackage[title]{appendix}

\newcommand{\mailtourl}[1]{\href{mailto:#1}{\tt #1}}
\newcommand{\tagvalue}[1]{\tt #1}
\newcommand{\tagregex}[1]{\tt #1}

\begin{document}

\documentclass[10pt]{article}
\usepackage[margin=1in]{geometry}
\usepackage{longtable}
\usepackage[pdfborder={0 0 0},hyperfootnotes=false]{hyperref}
\usepackage[title]{appendix}

\newcommand{\rulename}[1]{\tt #1}
\newcommand{\rulecategory}[1]{\tt #1}
\newcommand{\samrule}{\tt SAM}
\newcommand{\v15}{\tt v1.5}
\newcommand{\vcf43}{\tt VCFv4.3}
% #1: error message
% #2: rule description
% #3: categories
\newcommand{\samstrictrule}[3]{
#
	\paragraph{} #3
	% error message formatting
	{\tt #1}
	% 
	#2
}
% #1: header
% #2: categories
\newcommand{\headerrequired}[2]{
	\samstrictrule{Missing #1 header}{A #1 header must be present.}{#2}
}\
\newcommand{\headerunique}[2]{
	\samstrictrule{Only one #1 header may be present}{Multiple #1 headers must not be present.}{#2}
}
% #1: header
% #2: tag
% #3: categories
\newcommand{\headertagrequired}[3]{
	\samstrictrule{Missing #1 header #2 tag}{The #1 header #2 must be present.}{#3}
}
% #1: header
% #2: tag
% #4: categories
\newcommand{\headertagunique}[3]{
	\samstrictrule{Duplicate #1 header #2 tags.}{Each #1 header #2 tags must be unique.}{#2}
}
% #1: header
% #2: tag
% #3: regex
% #4: categories
\newcommand{\headertagregex}[4]{
	\samstrictrule{Malformed #1 header #2 tag}{The #1 header #2 tag must conform to the regex {\tt #3}}{#4}
}
% #1: header
% #2: tag
% #3: possible tag values
% #4: categories
\newcommand{\headertagvalues}[4]{
	\samstrictrule{Malformed #1 header #2 tag}{The #1 header #2 tag must contain one of #3}{#4}
}
% #1: header
% #2: tag
% #3: categories
\newcommand{\headertagmatchsamspecs}[3]{
	\samstrictrule{Malformed #1 header #2 tag}{The #1 header #2 tag must match the format defined defined in the SAM specifications.}{#3}
}

\begin{document}

\documentclass[10pt]{article}
\usepackage[margin=1in]{geometry}
\usepackage{longtable}
\usepackage[pdfborder={0 0 0},hyperfootnotes=false]{hyperref}
\usepackage[title]{appendix}

\newcommand{\rulename}[1]{\tt #1}
\newcommand{\rulecategory}[1]{\tt #1}
\newcommand{\samrule}{\tt SAM}
\newcommand{\v15}{\tt v1.5}
\newcommand{\vcf43}{\tt VCFv4.3}
% #1: error message
% #2: rule description
% #3: categories
\newcommand{\samstrictrule}[3]{
#
	\paragraph{} #3
	% error message formatting
	{\tt #1}
	% 
	#2
}
% #1: header
% #2: categories
\newcommand{\headerrequired}[2]{
	\samstrictrule{Missing #1 header}{A #1 header must be present.}{#2}
}\
\newcommand{\headerunique}[2]{
	\samstrictrule{Only one #1 header may be present}{Multiple #1 headers must not be present.}{#2}
}
% #1: header
% #2: tag
% #3: categories
\newcommand{\headertagrequired}[3]{
	\samstrictrule{Missing #1 header #2 tag}{The #1 header #2 must be present.}{#3}
}
% #1: header
% #2: tag
% #4: categories
\newcommand{\headertagunique}[3]{
	\samstrictrule{Duplicate #1 header #2 tags.}{Each #1 header #2 tags must be unique.}{#2}
}
% #1: header
% #2: tag
% #3: regex
% #4: categories
\newcommand{\headertagregex}[4]{
	\samstrictrule{Malformed #1 header #2 tag}{The #1 header #2 tag must conform to the regex {\tt #3}}{#4}
}
% #1: header
% #2: tag
% #3: possible tag values
% #4: categories
\newcommand{\headertagvalues}[4]{
	\samstrictrule{Malformed #1 header #2 tag}{The #1 header #2 tag must contain one of #3}{#4}
}
% #1: header
% #2: tag
% #3: categories
\newcommand{\headertagmatchsamspecs}[3]{
	\samstrictrule{Malformed #1 header #2 tag}{The #1 header #2 tag must match the format defined defined in the SAM specifications.}{#3}
}

\begin{document}

\documentclass[10pt]{article}
\usepackage[margin=1in]{geometry}
\usepackage{longtable}
\usepackage[pdfborder={0 0 0},hyperfootnotes=false]{hyperref}
\usepackage[title]{appendix}

\newcommand{\rulename}[1]{\tt #1}
\newcommand{\rulecategory}[1]{\tt #1}
\newcommand{\samrule}{\tt SAM}
\newcommand{\v15}{\tt v1.5}
\newcommand{\vcf43}{\tt VCFv4.3}
% #1: error message
% #2: rule description
% #3: categories
\newcommand{\samstrictrule}[3]{
#
	\paragraph{} #3
	% error message formatting
	{\tt #1}
	% 
	#2
}
% #1: header
% #2: categories
\newcommand{\headerrequired}[2]{
	\samstrictrule{Missing #1 header}{A #1 header must be present.}{#2}
}\
\newcommand{\headerunique}[2]{
	\samstrictrule{Only one #1 header may be present}{Multiple #1 headers must not be present.}{#2}
}
% #1: header
% #2: tag
% #3: categories
\newcommand{\headertagrequired}[3]{
	\samstrictrule{Missing #1 header #2 tag}{The #1 header #2 must be present.}{#3}
}
% #1: header
% #2: tag
% #4: categories
\newcommand{\headertagunique}[3]{
	\samstrictrule{Duplicate #1 header #2 tags.}{Each #1 header #2 tags must be unique.}{#2}
}
% #1: header
% #2: tag
% #3: regex
% #4: categories
\newcommand{\headertagregex}[4]{
	\samstrictrule{Malformed #1 header #2 tag}{The #1 header #2 tag must conform to the regex {\tt #3}}{#4}
}
% #1: header
% #2: tag
% #3: possible tag values
% #4: categories
\newcommand{\headertagvalues}[4]{
	\samstrictrule{Malformed #1 header #2 tag}{The #1 header #2 tag must contain one of #3}{#4}
}
% #1: header
% #2: tag
% #3: categories
\newcommand{\headertagmatchsamspecs}[3]{
	\samstrictrule{Malformed #1 header #2 tag}{The #1 header #2 tag must match the format defined defined in the SAM specifications.}{#3}
}

\begin{document}

\input{SAMstrict.ver}
\title{SAM Strict Specification}
\author{Daniel L Cameron}
\date{\headdate}
\maketitle
\begin{quote}\small
The master version of this document can be found at
\url{https://github.com/samtools/hts-specs}.\\
This printing is version~\commitdesc\ from that repository,
last modified on the date shown above.
\end{quote}
\vspace*{1em}

\noindent
This document is a companion to the {\sl Sequence Alignment/Map Format
Specification} that defines the SAM file format.\footnote{See
\href{http://samtools.github.io/hts-specs/SAMv1.pdf}{\tt SAMv1.pdf} at \url{https://github.com/samtools/hts-specs}.}
The SAM file format defines the syntax required for a file to be
a valid SAM file. It does not require such files to be semantically
valid and internally consistent.
This document describes a set of additional semantic restrictions
for which the subset of syntactically valid SAM files that comply
with these restrictions can be described as \textit{SAM strict
compliant}.

\renewcommand{\abstractname}{Introduction}
\begin{abstract}

The SAM specifications have been instrumental in standardising
the file formats used for sequence alignment. A large ecosystem of
bioinformatics tools is now capable of reading and/or writing
SAM files. Unfortunately, many tools that read SAM files are tightly
coupled to a particular upstream tool 
and fail to correctly execute on valid SAM files written by other
tools. In part, this is due to the lack of semantic restrictions
inherent in the SAM file format. A syntactically valid SAM file
can be both internally inconsistent and semantically nonsensical.

The purpose of this document is to provide a baseline of semantic
validity for which tools should comply with when outputing SAM
files, and tools which input SAM files can safely assume when
they require input files to be \textit{SAM strict compliant}.

\end{abstract}

\section{Headers}
\samstrictrule{Undefined reserved header present}{Upper-case header record type codes are not defined in the SAM specifications must not be used.}
\samstrictrule{Undefined header tag present}{Upper-case header tags not defined in the SAM specifications must not be used.}
\samstrictrule{Tag present as both lowercase and uppercase}{A file should not contain the same tag in both upper-case and lowercase format. See the SAM specifications header tags best practice footnote.}
\samstrictrule{Malformed header}{Header lines with conform to either the {\tt
  /\char94@[A-Z][A-Z](\char92t[A-Za-z][A-Za-z0-9]:[
  -\char126]+)+\$/} or {\tt /\char94@CO\char92t.*/} or {\tt /\char94@CO\char92t.*/} regex.}{\samrule}

\subsection{HD}
\headerrequired{HD}
\headerunique{HD}
\headertagrequired{HD}{VN}{\samrule}
\samstrictrule{File does not start with HD header.}{The first header defined must be the HD header.}{\samrule}
\headertagregex{HD}{VN}{/\char94[0-9]+\char92.[0-9]+\$/}{\samrule}
\samstrictrule{Unknown SAM version}{The HD header VN tag version number must match a published version of the SAM specifications.}
\headertagvalues{HD}{SO}{{\tt unknown}, {\tt unsorted}, {\tt queryname} and {\tt coordinate}}{\samrule}
\headertagvalues{HD}{GO}{{\tt none}, {\tt query}, {\tt reference}}{\samrule}
\samstrictrule{Inconsistent HD header SO and GO tags}{The record orderings defined in the HD header SO and GO tags must be consistent}

\subsection{SQ}
\headertagrequired{SQ}{SN}{\samrule}
\headertagregex{SQ}{SN}{[!-)+-\char60\char62-\char126][!-\char126]*}{\samrule}
\headertagunique{SQ}{SN}
\headertagrequired{SQ}{LN}{\samrule}
\samstrictrule{Malformed SQ header LN tag}{The SQ header LN tag value must be an integer.}{\samrule}
\samstrictrule{Unsupported reference sequence length}{The SQ header LN tag value must greater than zero and less than 2147483648}{\samrule}
\headertagrequired{SQ}{M5}{\samrule}
\headertagmatchsamspecs{SQ}{AH}{\samrule}
\samstrictrule{Alternate locus references unknown reference sequence name}{Sequence names in the SQ header AH tag must match a SQ header SN reference sequence name.}{\samrule}
\headertagmatchsamspecs{SQ}{AN}{\samrule}
\samstrictrule{Duplicate alternative reference sequence names.}{Alternative reference sequence names defined in SQ header AN tags must be unique. A single tag cannot define cannot define the same alternative reference sequence name multiple times and multiple SQ headers cannot define the same alternative reference sequence name.}{\samrule}
\samstrictrule{Invalid alternative reference sequence names.}{Sequence names in the SQ header AN tag must not match any SQ header SN reference sequence names. }{\samrule}
\headertagregex{SQ}{M5}{[a-f0-9]\{32\}}

\subsection{RG}
\headertagrequired{RQ}{ID}{\samrule}
\headertagunique{RQ}{ID}{\samrule}
\samstrictrule{RQ header DT tag is not ISO8601}{RQ header DT tag must contain a valid date in ISO8601 format}{\samrule}
\headertagregex{RQ}{FO}{/\char92*|[ACMGRSVTWYHKDBN]+/}{\samrule}
\samstrictrule{Malformed RQ header PI tag}{The RQ header PI tag value must be a floating point value.}{\samrule}
\headertagvalues{RG}{PL}{{\tt CAPILLARY}, {\tt LS454}, {\tt ILLUMINA}, {\tt SOLID}, {\tt HELICOS}, {\tt IONTORRENT}, {\tt ONT}, and {\tt PACBIO}}{\samrule}
\samstrictrule{Invalid RG program group}{The RG header PG tag must contain one of the program groups specified in an ID tag of a PG header.}

\subsection{PG}
\headertagrequired{PG}{ID}{\samrule}
\headertagunique{PG}{ID}{\samrule}
\samstrictrule{Invalid PG header PP tag}{The PG header PP tag must contain one of the program groups specified in an ID tag of a PG header.}

\section{General}

\subsection{File Format}

\samstrictrule{File is not UTF-8}{The file must use UTF-8 encoding.}{\samrule}
\samstrictrule{Inconsistent line terminators}{All lines must be separated with the same new line character\(s\).}
\samstrictrule{Malformed floating point value}{All floating point values must conform to the regex {\tt [-+]?[0-9]*\char92.?[0-9]+([eE][-+]?[0-9]+)?}}
\samstrictrule{Malformed integer value}{All integer values must conform to the regex [-+]?[0-9]+}}

\subsection{Ordering}

\samstrictrule{Record ordering does not match HD header SO tag}{The order of records must be consistent with the HD header SO tag}
\samstrictrule{Record ordering does not match HD header GO tag}{The order of records must be consistent with the HD header GO tag}
\samstrictrule{Orphaned unmapped read}{If a read is unmapped, RNAME and POS must either be * and 0, or the RNAME and POS of another read from the same template.}

\section{Records}

\subsection{QNAME}
\samstrictrule{Empty QNAME}{QNAME fields must be at least one character in length}{\samrule}
\samstrictrule{QNAME too long}{QNAME fields must be at less than 255 characters in length}{\samrule}
\samstrictrule{Invalid character in QNAME}{QNAME fields must conform to the regex {TT \verb:[!-?A-~]}}{\samrule}

\subsection{FLAG}
\samstrictrule{Incorrect FLAG 0x1}{All templates with multiple segments must have FLAG 0x1 set}
\samstrictrule{Incorrect FLAG 0x1}{All templates with a single segment must not have FLAG 0x1 set}
\samstrictrule{Inconsistent FLAG 0x1}{All records with the same QNAME must have the same FLAG 0x1 value}
\samstrictrule{Inconsistent FLAG 0x2}{All primary records with the same QNAME must have the same FLAG 0x2 value}
\samstrictrule{Missing primary alignment record}{No supplementary or secondary alignments may exist for reads with an unmapped with primary alignment.}{\samrule}
\samstrictrule{Inconsistent FLAGs 0x1 0x2}{The 0x2 FLAG must not be set if the 0x1 FLAG is not.}
\samstrictrule{Inconsistent FLAGs 0x1 0x8}{The 0x8 FLAG must not be set if the 0x1 FLAG is not.}
\samstrictrule{Inconsistent FLAGs 0x1 0x20}{The 0x20 FLAG must not be set if the 0x1 FLAG is not.}
\samstrictrule{Inconsistent FLAGs 0x1 0x40}{The 0x40 FLAG must not be set if the 0x1 FLAG is not.}
\samstrictrule{Inconsistent FLAGs 0x1 0x80}{The 0x80 FLAG must not be set if the 0x1 FLAG is not.}
\samstrictrule{Inconsistent FLAGs 0x2 0x4}{The 0x2 FLAG must not be set if 0x4 is set in any primary alignments in the template}
\samstrictrule{Inconsistent FLAGs 0x4 0x8}{The 0x8 FLAG for primary record for each segment must match the 0x4 FLAG of the primary record for the next segment in the template}
\samstrictrule{Inconsistent FLAG 0x10 0x20}{The 0x20 FLAG must match the 0x10 FLAG for the primary alignment of the next segment in the template}
\samstrictrule{FLAG 0x20 set on unmapped read}{The 0x10 FLAG must not be set if the 0x4 FLAG is is set.}
\samstrictrule{Duplicate first segment primary records}{Of the primary records with the same QNAME, at most one can have FLAG 0x40 set}
\samstrictrule{Missing first segment primary record}{Of the records with the same QNAME with 0x1 FLAG, at least one record must have FLAG 0x40 set.}
\samstrictrule{Duplicate last segment primary records}{Of the primary records with the same QNAME, at most one can have FLAG 0x80 set}
\samstrictrule{Missing last segment primary record}{Of the records with the same QNAME with 0x1 FLAG, at least one record must have FLAG 0x80 set.}
\samstrictrule{Multiple primary alignment records}{Each segment must have at most one record with FLAG 0x100 and 0x800 not set.}{\samrule}
\samstrictrule{Missing primary alignment record}{Each segment must have at least one record with FLAG 0x100 and 0x800 not set.}{\samrule}
\samstrictrule{Unknown FLAG bit set}{FLAG bits higher than 0x800 must not be set}{\samrule}

\subsection{RNAME}
\samstrictrule{Malformed RNAME}{RNAME must conform to the regex {\tt \char92*|[!-()+-\char60\char62-\char126][!-\char126]*}}{\samrule}
\samstrictrule{RNAME not present in reference}{RNAME must be equal to the value of one of the SQ SN values defined in the header.}
\samstrictrule{RNAME contains character not supported by VCFv4.3}{RNAME must not contain any of the following characters: {\tt \char60\char62\char91\char93\char58\char42}}{\vcf43}
\samstrictrule{RNAME not supported by VCFv4.3}{RNAME must be not be one of {\tt DEL}, {\tt INS}, {\tt DUP}, {\tt INV}, {\tt CNV}, or {\tt BND}.}{\vcf43}

\subsection{POS}
\samstrictrule{Record placed outside of reference sequence}{If FLAG 0x4 is not set, POS must be between 0 and the length of the RNAME reference sequence inclusive. The length of the RNAME reference sequence can be found in the SQ header LN tag value for the SQ header with a SN tag matching the RNAME.}
\samstrictrule{Invalid POS}{POS cannot be 0 if FLAG 0x4 is set.}
\samstrictrule{Invalid POS}{POS cannot be negative.}{\samrule}
\samstrictrule{Invalid POS}{POS cannot be greater than 2147483647.}{\samrule}
\samstrictrule{POS specified without RNAME}{If RNAME is *, POS must be 0.}

\subsection{MAPQ}

\subsection{CIGAR}
\samstrictrule{Invalid CIGAR}{All CIGAR strings must conform to the regex {\tt \char92*|([0-9]+[MIDNSHPX=])+}}{\samrule}
\samstrictrule{Empty CIGAR}{All CIGAR strings must have at least one CIGAR operator}
\samstrictrule{Zero length CIGAR operator}{All CIGAR operators must have a non-zero positive length}
\samstrictrule{CIGAR contains operator repeat}{All adjacent CIGAR operators must be different.}
\samstrictrule{CIGAR does not contain any mapped bases}{All CIGARs must include a CIGAR operator that consumes a reference base.}{\samrule}
	{\tt Should we allow alignments with zero mapped bases? Seven bridges has a graph-based aligner that will
emit CIGARs such as 100I for alignments that align to a known insertion that is not included in the reference. Useful for local assembly but technical voilates the SAM specifications}
\samstrictrule{Incorrect CIGAR length}{Sum of lengths of the M/I/S/=/X operations must equal the length of SEQ when both CIGAR and SEQ are available.}{\samrule}
\samstrictrule{Invalid CIGAR hard clip}{H must only be present as the first and/or last operation.}{\samrule}
\samstrictrule{Invalid CIGAR soft clip}{S must only have H operations between them and the ends of the CIGAR string.}{\samrule}
\samstrictrule{CIGAR overhangs reference sequence}{POS plus the number of reference bases consumed by the CIGAR must not exceed the length of the RNAME reference sequence.}
\samstrictrule{Inconsistent CIGAR read lengths}{All mapped alignments for a given segment must have matching read lengths. That is, the sum of lengths of the M/I/S/=/X/H operations must be equal.}

\samstrictrule{Unusual indel positioning}{TODO: should we disallow I/D operators at the ends of reads? There was some ambiguity as to how deletions interacted with POS but I think the spec has been updated in favour of the BWA interpretation since that discussion.}

\subsection{RNEXT}
\samstrictrule{Invalid RNEXT}{If the template contains one segment RNEXT must be *}
\samstrictrule{RNEXT not present in reference}{RNEXT must be equal to the value of one of the SQ SN values defined in the header.}
\samstrictrule{Invalid RNEXT}{If the primary alignment of the next read in the template is mapped, RNEXT must not be {\tt *}}{\samrule}
\samstrictrule{RNEXT not using =}{If the primary alignment of the next read in the template is mapped to the same reference sequence, RNEXT must be {\tt =}}{\samrule}
\samstrictrule{Incorrect RNEXT}{If this read is unmapped or the primary alignment of the next read in the template is mapped to the a different reference sequence, RNEXT must match the RNAME of the next read.}

\subsection{PNEXT}
\samstrictrule{Invalid PNEXT}{If the template contains one segment PNEXT must be 0}
\samstrictrule{Incorrect PNEXT}{If the primary alignment of the next read in the template is unmapped, PNEXT must be 0}
\samstrictrule{Incorrect PNEXT}{If the primary alignment of the next read in the template is mapped, PNEXT must match the POS of that record.}

\subsection{TLEN}
TODO: can we actually do anything for this?

\subsection{SEQ}
\samstrictrule{Inconsistent SEQ read lengths}{All alignments of a given segment must have consistent SEQ lengths. That is, for all non-* SEQ, SEQ + length of CIGAR hard clip must be equal. }
\samstrictrule{Inconsistent SEQ sequences}{All alignments of a given segment must have consistent base calls. A base cannot be called an A in one record, but a T in another. Note that to determine the read base, both the 0x10 FLAG, and any hard clipping CIGAR operators need to be taken into account.}

TODO: I am up to here

\subsection{QUAL}















































\paragraph{}


\paragraph{}




\subsection{TODO: categorize}

- secondary alignment mate info must match primary
- supplementary alignment mate info must match primary

\paragraph{}

All read alignments must have CIGARs with matching read length. That is,
the sum of lengths of the M/I/S/=/X operations must be equal for all mapped read alignment.
This means that chimeric alignments must include either soft clipping or
hard clipping CIGAR operations for read bases which were not aligned.

\paragraph{}

All read alignments with SEQ not equal to * must have a SEQ consistent
with all other read alignments. That is, chimeric and secondary alignments that have non-* SEQ
must be consistent with all other records defining a segment sequence. For
example, 10th base in a read cannot be be A in one alignment, but T in another.

\paragraph{}

All read alignments with QUAL not equal to * must have a SEQ consistent
with all other read alignments.

\subsection{Mate alignments}

\paragraph{}

Unmapped reads must have RNAME and POS identical to that of the primary
non-supplementary alignment of first mapped read from the originating template.
For paired-end or mate-pair sequencing, this equivalent
to setting the RNAME/POS of unmapped reads to the RNAME/POS of the mate.

\paragraph{}

For templates with multiple reads, RNEXT and PNEXT must match the
alignment of the non-supplementary primary alignment of the next read.
As per the SAM specifications, for the last read, the next read is the first read in the template.

\paragraph{}

If all segments in a template are unmapped, RNAME must be set as `*' and POS as 0.

\subsection{Reference bounds}

\paragraph{}

Read alignments must not extend past the start or end of the aligned RNAME.

\subsection{Mapping Qualities}

\paragraph{}

All mapping quality scores, including those defined in tags must be within the range [0, 255].
A value 255 indicates that the mapping quality is not available and must only be used if the
mapping quality field is required. For example, a mapping quality field value is required for
MAPQ field and the mapq portion of the SA tag, but as the AM is optional, a mapping quality
field value is not required and the AM tag should be omitted entirely if a mapping quality is
not available.

\section{SAM Tags}

\subsection{Standard Tags}

\paragraph{}

No record can include any reserved tags not defined in the
{\sl Sequence Alignment/Map Optional Fields Specification}.
Non-standard tags must start X, Y, Z or a lowercase letter as per the SAM specifications.

\paragraph{}

The \textit{type} of all \textit{standard tags} must match the type
defined in the {\sl Sequence Alignment/Map Optional Fields Specification}.

\paragraph{}

All tag values must be consistent with the format
defined in the {\sl Sequence Alignment/Map Optional Fields Specification}.

\subsection{SA}

For the purpose of this section, a \textit{SA record set} is a set of SAM records
from a single \textit{read} which collectively represent a single \textit{chimeric alignment}.

\samstrictrule{Missing SA tag}{All records with FLAG 0x800 set must have a SA tag defined.}{\samrule}
\samstrictrule{Missing non-supplementary chimeric alignment record}{Each chimeric alignment must have a record with FLAG 0x800 not set.}

\paragraph{}

All SA tag values must satisfy the SA tag regular expression
defined in the {\sl Sequence Alignment/Map Optional Fields Specification}.

\paragraph{}

All records referenced in the SA tag of a given record must have a SA tag defined.

\paragraph{}

All records referenced in the SA tag of a given record must include the given
record in their SA tag.

\paragraph{}

All records in a \textit{SA record set} with a FI tag defined must have the same FI tag value.

TODO: which other tags?

\paragraph{}

All records referenced in the SA tag must exist with matching \textit{rname, pos, strand, CIGAR}.

\paragraph{}

The SA \textit{mapq} of all references to a given record in a \textit{SA record set} must
match the record mapping quality.

\paragraph{}

The SA \textit{NM} of all references to a given record in a \textit{SA record set} must
match the record \textit{NM} tag value.

\paragraph{}

All records except 1 in a \textit{SA record set} must have the 0x800 (supplementary alignment) FLAG bit set.

\paragraph{}

The first SA record in all supplementary alignment records must be the canonical non-supplementary alignment.

\paragraph{}

All records in a \textit{SA record set} must have the same 0x100 (secondary alignment) FLAG value.

\paragraph{}

All records in a \textit{SA record set} must have FLAG bit 0x4 (segment unmapped) not set.

\paragraph{}

All records in a \textit{SA record set} must align at least one read base that does not
overlap with any other alignments in the \textit{SA record set}.
That is, a chimeric alignment cannot contain superfluous alignment records.



- SEQ must match CIGAR
- Read lengths must be consistent


\end{document}

\title{SAM Strict Specification}
\author{Daniel L Cameron}
\date{\headdate}
\maketitle
\begin{quote}\small
The master version of this document can be found at
\url{https://github.com/samtools/hts-specs}.\\
This printing is version~\commitdesc\ from that repository,
last modified on the date shown above.
\end{quote}
\vspace*{1em}

\noindent
This document is a companion to the {\sl Sequence Alignment/Map Format
Specification} that defines the SAM file format.\footnote{See
\href{http://samtools.github.io/hts-specs/SAMv1.pdf}{\tt SAMv1.pdf} at \url{https://github.com/samtools/hts-specs}.}
The SAM file format defines the syntax required for a file to be
a valid SAM file. It does not require such files to be semantically
valid and internally consistent.
This document describes a set of additional semantic restrictions
for which the subset of syntactically valid SAM files that comply
with these restrictions can be described as \textit{SAM strict
compliant}.

\renewcommand{\abstractname}{Introduction}
\begin{abstract}

The SAM specifications have been instrumental in standardising
the file formats used for sequence alignment. A large ecosystem of
bioinformatics tools is now capable of reading and/or writing
SAM files. Unfortunately, many tools that read SAM files are tightly
coupled to a particular upstream tool 
and fail to correctly execute on valid SAM files written by other
tools. In part, this is due to the lack of semantic restrictions
inherent in the SAM file format. A syntactically valid SAM file
can be both internally inconsistent and semantically nonsensical.

The purpose of this document is to provide a baseline of semantic
validity for which tools should comply with when outputing SAM
files, and tools which input SAM files can safely assume when
they require input files to be \textit{SAM strict compliant}.

\end{abstract}

\section{Headers}
\samstrictrule{Undefined reserved header present}{Upper-case header record type codes are not defined in the SAM specifications must not be used.}
\samstrictrule{Undefined header tag present}{Upper-case header tags not defined in the SAM specifications must not be used.}
\samstrictrule{Tag present as both lowercase and uppercase}{A file should not contain the same tag in both upper-case and lowercase format. See the SAM specifications header tags best practice footnote.}
\samstrictrule{Malformed header}{Header lines with conform to either the {\tt
  /\char94@[A-Z][A-Z](\char92t[A-Za-z][A-Za-z0-9]:[
  -\char126]+)+\$/} or {\tt /\char94@CO\char92t.*/} or {\tt /\char94@CO\char92t.*/} regex.}{\samrule}

\subsection{HD}
\headerrequired{HD}
\headerunique{HD}
\headertagrequired{HD}{VN}{\samrule}
\samstrictrule{File does not start with HD header.}{The first header defined must be the HD header.}{\samrule}
\headertagregex{HD}{VN}{/\char94[0-9]+\char92.[0-9]+\$/}{\samrule}
\samstrictrule{Unknown SAM version}{The HD header VN tag version number must match a published version of the SAM specifications.}
\headertagvalues{HD}{SO}{{\tt unknown}, {\tt unsorted}, {\tt queryname} and {\tt coordinate}}{\samrule}
\headertagvalues{HD}{GO}{{\tt none}, {\tt query}, {\tt reference}}{\samrule}
\samstrictrule{Inconsistent HD header SO and GO tags}{The record orderings defined in the HD header SO and GO tags must be consistent}

\subsection{SQ}
\headertagrequired{SQ}{SN}{\samrule}
\headertagregex{SQ}{SN}{[!-)+-\char60\char62-\char126][!-\char126]*}{\samrule}
\headertagunique{SQ}{SN}
\headertagrequired{SQ}{LN}{\samrule}
\samstrictrule{Malformed SQ header LN tag}{The SQ header LN tag value must be an integer.}{\samrule}
\samstrictrule{Unsupported reference sequence length}{The SQ header LN tag value must greater than zero and less than 2147483648}{\samrule}
\headertagrequired{SQ}{M5}{\samrule}
\headertagmatchsamspecs{SQ}{AH}{\samrule}
\samstrictrule{Alternate locus references unknown reference sequence name}{Sequence names in the SQ header AH tag must match a SQ header SN reference sequence name.}{\samrule}
\headertagmatchsamspecs{SQ}{AN}{\samrule}
\samstrictrule{Duplicate alternative reference sequence names.}{Alternative reference sequence names defined in SQ header AN tags must be unique. A single tag cannot define cannot define the same alternative reference sequence name multiple times and multiple SQ headers cannot define the same alternative reference sequence name.}{\samrule}
\samstrictrule{Invalid alternative reference sequence names.}{Sequence names in the SQ header AN tag must not match any SQ header SN reference sequence names. }{\samrule}
\headertagregex{SQ}{M5}{[a-f0-9]\{32\}}

\subsection{RG}
\headertagrequired{RQ}{ID}{\samrule}
\headertagunique{RQ}{ID}{\samrule}
\samstrictrule{RQ header DT tag is not ISO8601}{RQ header DT tag must contain a valid date in ISO8601 format}{\samrule}
\headertagregex{RQ}{FO}{/\char92*|[ACMGRSVTWYHKDBN]+/}{\samrule}
\samstrictrule{Malformed RQ header PI tag}{The RQ header PI tag value must be a floating point value.}{\samrule}
\headertagvalues{RG}{PL}{{\tt CAPILLARY}, {\tt LS454}, {\tt ILLUMINA}, {\tt SOLID}, {\tt HELICOS}, {\tt IONTORRENT}, {\tt ONT}, and {\tt PACBIO}}{\samrule}
\samstrictrule{Invalid RG program group}{The RG header PG tag must contain one of the program groups specified in an ID tag of a PG header.}

\subsection{PG}
\headertagrequired{PG}{ID}{\samrule}
\headertagunique{PG}{ID}{\samrule}
\samstrictrule{Invalid PG header PP tag}{The PG header PP tag must contain one of the program groups specified in an ID tag of a PG header.}

\section{General}

\subsection{File Format}

\samstrictrule{File is not UTF-8}{The file must use UTF-8 encoding.}{\samrule}
\samstrictrule{Inconsistent line terminators}{All lines must be separated with the same new line character\(s\).}
\samstrictrule{Malformed floating point value}{All floating point values must conform to the regex {\tt [-+]?[0-9]*\char92.?[0-9]+([eE][-+]?[0-9]+)?}}
\samstrictrule{Malformed integer value}{All integer values must conform to the regex [-+]?[0-9]+}}

\subsection{Ordering}

\samstrictrule{Record ordering does not match HD header SO tag}{The order of records must be consistent with the HD header SO tag}
\samstrictrule{Record ordering does not match HD header GO tag}{The order of records must be consistent with the HD header GO tag}
\samstrictrule{Orphaned unmapped read}{If a read is unmapped, RNAME and POS must either be * and 0, or the RNAME and POS of another read from the same template.}

\section{Records}

\subsection{QNAME}
\samstrictrule{Empty QNAME}{QNAME fields must be at least one character in length}{\samrule}
\samstrictrule{QNAME too long}{QNAME fields must be at less than 255 characters in length}{\samrule}
\samstrictrule{Invalid character in QNAME}{QNAME fields must conform to the regex {TT \verb:[!-?A-~]}}{\samrule}

\subsection{FLAG}
\samstrictrule{Incorrect FLAG 0x1}{All templates with multiple segments must have FLAG 0x1 set}
\samstrictrule{Incorrect FLAG 0x1}{All templates with a single segment must not have FLAG 0x1 set}
\samstrictrule{Inconsistent FLAG 0x1}{All records with the same QNAME must have the same FLAG 0x1 value}
\samstrictrule{Inconsistent FLAG 0x2}{All primary records with the same QNAME must have the same FLAG 0x2 value}
\samstrictrule{Missing primary alignment record}{No supplementary or secondary alignments may exist for reads with an unmapped with primary alignment.}{\samrule}
\samstrictrule{Inconsistent FLAGs 0x1 0x2}{The 0x2 FLAG must not be set if the 0x1 FLAG is not.}
\samstrictrule{Inconsistent FLAGs 0x1 0x8}{The 0x8 FLAG must not be set if the 0x1 FLAG is not.}
\samstrictrule{Inconsistent FLAGs 0x1 0x20}{The 0x20 FLAG must not be set if the 0x1 FLAG is not.}
\samstrictrule{Inconsistent FLAGs 0x1 0x40}{The 0x40 FLAG must not be set if the 0x1 FLAG is not.}
\samstrictrule{Inconsistent FLAGs 0x1 0x80}{The 0x80 FLAG must not be set if the 0x1 FLAG is not.}
\samstrictrule{Inconsistent FLAGs 0x2 0x4}{The 0x2 FLAG must not be set if 0x4 is set in any primary alignments in the template}
\samstrictrule{Inconsistent FLAGs 0x4 0x8}{The 0x8 FLAG for primary record for each segment must match the 0x4 FLAG of the primary record for the next segment in the template}
\samstrictrule{Inconsistent FLAG 0x10 0x20}{The 0x20 FLAG must match the 0x10 FLAG for the primary alignment of the next segment in the template}
\samstrictrule{FLAG 0x20 set on unmapped read}{The 0x10 FLAG must not be set if the 0x4 FLAG is is set.}
\samstrictrule{Duplicate first segment primary records}{Of the primary records with the same QNAME, at most one can have FLAG 0x40 set}
\samstrictrule{Missing first segment primary record}{Of the records with the same QNAME with 0x1 FLAG, at least one record must have FLAG 0x40 set.}
\samstrictrule{Duplicate last segment primary records}{Of the primary records with the same QNAME, at most one can have FLAG 0x80 set}
\samstrictrule{Missing last segment primary record}{Of the records with the same QNAME with 0x1 FLAG, at least one record must have FLAG 0x80 set.}
\samstrictrule{Multiple primary alignment records}{Each segment must have at most one record with FLAG 0x100 and 0x800 not set.}{\samrule}
\samstrictrule{Missing primary alignment record}{Each segment must have at least one record with FLAG 0x100 and 0x800 not set.}{\samrule}
\samstrictrule{Unknown FLAG bit set}{FLAG bits higher than 0x800 must not be set}{\samrule}

\subsection{RNAME}
\samstrictrule{Malformed RNAME}{RNAME must conform to the regex {\tt \char92*|[!-()+-\char60\char62-\char126][!-\char126]*}}{\samrule}
\samstrictrule{RNAME not present in reference}{RNAME must be equal to the value of one of the SQ SN values defined in the header.}
\samstrictrule{RNAME contains character not supported by VCFv4.3}{RNAME must not contain any of the following characters: {\tt \char60\char62\char91\char93\char58\char42}}{\vcf43}
\samstrictrule{RNAME not supported by VCFv4.3}{RNAME must be not be one of {\tt DEL}, {\tt INS}, {\tt DUP}, {\tt INV}, {\tt CNV}, or {\tt BND}.}{\vcf43}

\subsection{POS}
\samstrictrule{Record placed outside of reference sequence}{If FLAG 0x4 is not set, POS must be between 0 and the length of the RNAME reference sequence inclusive. The length of the RNAME reference sequence can be found in the SQ header LN tag value for the SQ header with a SN tag matching the RNAME.}
\samstrictrule{Invalid POS}{POS cannot be 0 if FLAG 0x4 is set.}
\samstrictrule{Invalid POS}{POS cannot be negative.}{\samrule}
\samstrictrule{Invalid POS}{POS cannot be greater than 2147483647.}{\samrule}
\samstrictrule{POS specified without RNAME}{If RNAME is *, POS must be 0.}

\subsection{MAPQ}

\subsection{CIGAR}
\samstrictrule{Invalid CIGAR}{All CIGAR strings must conform to the regex {\tt \char92*|([0-9]+[MIDNSHPX=])+}}{\samrule}
\samstrictrule{Empty CIGAR}{All CIGAR strings must have at least one CIGAR operator}
\samstrictrule{Zero length CIGAR operator}{All CIGAR operators must have a non-zero positive length}
\samstrictrule{CIGAR contains operator repeat}{All adjacent CIGAR operators must be different.}
\samstrictrule{CIGAR does not contain any mapped bases}{All CIGARs must include a CIGAR operator that consumes a reference base.}{\samrule}
	{\tt Should we allow alignments with zero mapped bases? Seven bridges has a graph-based aligner that will
emit CIGARs such as 100I for alignments that align to a known insertion that is not included in the reference. Useful for local assembly but technical voilates the SAM specifications}
\samstrictrule{Incorrect CIGAR length}{Sum of lengths of the M/I/S/=/X operations must equal the length of SEQ when both CIGAR and SEQ are available.}{\samrule}
\samstrictrule{Invalid CIGAR hard clip}{H must only be present as the first and/or last operation.}{\samrule}
\samstrictrule{Invalid CIGAR soft clip}{S must only have H operations between them and the ends of the CIGAR string.}{\samrule}
\samstrictrule{CIGAR overhangs reference sequence}{POS plus the number of reference bases consumed by the CIGAR must not exceed the length of the RNAME reference sequence.}
\samstrictrule{Inconsistent CIGAR read lengths}{All mapped alignments for a given segment must have matching read lengths. That is, the sum of lengths of the M/I/S/=/X/H operations must be equal.}

\samstrictrule{Unusual indel positioning}{TODO: should we disallow I/D operators at the ends of reads? There was some ambiguity as to how deletions interacted with POS but I think the spec has been updated in favour of the BWA interpretation since that discussion.}

\subsection{RNEXT}
\samstrictrule{Invalid RNEXT}{If the template contains one segment RNEXT must be *}
\samstrictrule{RNEXT not present in reference}{RNEXT must be equal to the value of one of the SQ SN values defined in the header.}
\samstrictrule{Invalid RNEXT}{If the primary alignment of the next read in the template is mapped, RNEXT must not be {\tt *}}{\samrule}
\samstrictrule{RNEXT not using =}{If the primary alignment of the next read in the template is mapped to the same reference sequence, RNEXT must be {\tt =}}{\samrule}
\samstrictrule{Incorrect RNEXT}{If this read is unmapped or the primary alignment of the next read in the template is mapped to the a different reference sequence, RNEXT must match the RNAME of the next read.}

\subsection{PNEXT}
\samstrictrule{Invalid PNEXT}{If the template contains one segment PNEXT must be 0}
\samstrictrule{Incorrect PNEXT}{If the primary alignment of the next read in the template is unmapped, PNEXT must be 0}
\samstrictrule{Incorrect PNEXT}{If the primary alignment of the next read in the template is mapped, PNEXT must match the POS of that record.}

\subsection{TLEN}
TODO: can we actually do anything for this?

\subsection{SEQ}
\samstrictrule{Inconsistent SEQ read lengths}{All alignments of a given segment must have consistent SEQ lengths. That is, for all non-* SEQ, SEQ + length of CIGAR hard clip must be equal. }
\samstrictrule{Inconsistent SEQ sequences}{All alignments of a given segment must have consistent base calls. A base cannot be called an A in one record, but a T in another. Note that to determine the read base, both the 0x10 FLAG, and any hard clipping CIGAR operators need to be taken into account.}

TODO: I am up to here

\subsection{QUAL}















































\paragraph{}


\paragraph{}




\subsection{TODO: categorize}

- secondary alignment mate info must match primary
- supplementary alignment mate info must match primary

\paragraph{}

All read alignments must have CIGARs with matching read length. That is,
the sum of lengths of the M/I/S/=/X operations must be equal for all mapped read alignment.
This means that chimeric alignments must include either soft clipping or
hard clipping CIGAR operations for read bases which were not aligned.

\paragraph{}

All read alignments with SEQ not equal to * must have a SEQ consistent
with all other read alignments. That is, chimeric and secondary alignments that have non-* SEQ
must be consistent with all other records defining a segment sequence. For
example, 10th base in a read cannot be be A in one alignment, but T in another.

\paragraph{}

All read alignments with QUAL not equal to * must have a SEQ consistent
with all other read alignments.

\subsection{Mate alignments}

\paragraph{}

Unmapped reads must have RNAME and POS identical to that of the primary
non-supplementary alignment of first mapped read from the originating template.
For paired-end or mate-pair sequencing, this equivalent
to setting the RNAME/POS of unmapped reads to the RNAME/POS of the mate.

\paragraph{}

For templates with multiple reads, RNEXT and PNEXT must match the
alignment of the non-supplementary primary alignment of the next read.
As per the SAM specifications, for the last read, the next read is the first read in the template.

\paragraph{}

If all segments in a template are unmapped, RNAME must be set as `*' and POS as 0.

\subsection{Reference bounds}

\paragraph{}

Read alignments must not extend past the start or end of the aligned RNAME.

\subsection{Mapping Qualities}

\paragraph{}

All mapping quality scores, including those defined in tags must be within the range [0, 255].
A value 255 indicates that the mapping quality is not available and must only be used if the
mapping quality field is required. For example, a mapping quality field value is required for
MAPQ field and the mapq portion of the SA tag, but as the AM is optional, a mapping quality
field value is not required and the AM tag should be omitted entirely if a mapping quality is
not available.

\section{SAM Tags}

\subsection{Standard Tags}

\paragraph{}

No record can include any reserved tags not defined in the
{\sl Sequence Alignment/Map Optional Fields Specification}.
Non-standard tags must start X, Y, Z or a lowercase letter as per the SAM specifications.

\paragraph{}

The \textit{type} of all \textit{standard tags} must match the type
defined in the {\sl Sequence Alignment/Map Optional Fields Specification}.

\paragraph{}

All tag values must be consistent with the format
defined in the {\sl Sequence Alignment/Map Optional Fields Specification}.

\subsection{SA}

For the purpose of this section, a \textit{SA record set} is a set of SAM records
from a single \textit{read} which collectively represent a single \textit{chimeric alignment}.

\samstrictrule{Missing SA tag}{All records with FLAG 0x800 set must have a SA tag defined.}{\samrule}
\samstrictrule{Missing non-supplementary chimeric alignment record}{Each chimeric alignment must have a record with FLAG 0x800 not set.}

\paragraph{}

All SA tag values must satisfy the SA tag regular expression
defined in the {\sl Sequence Alignment/Map Optional Fields Specification}.

\paragraph{}

All records referenced in the SA tag of a given record must have a SA tag defined.

\paragraph{}

All records referenced in the SA tag of a given record must include the given
record in their SA tag.

\paragraph{}

All records in a \textit{SA record set} with a FI tag defined must have the same FI tag value.

TODO: which other tags?

\paragraph{}

All records referenced in the SA tag must exist with matching \textit{rname, pos, strand, CIGAR}.

\paragraph{}

The SA \textit{mapq} of all references to a given record in a \textit{SA record set} must
match the record mapping quality.

\paragraph{}

The SA \textit{NM} of all references to a given record in a \textit{SA record set} must
match the record \textit{NM} tag value.

\paragraph{}

All records except 1 in a \textit{SA record set} must have the 0x800 (supplementary alignment) FLAG bit set.

\paragraph{}

The first SA record in all supplementary alignment records must be the canonical non-supplementary alignment.

\paragraph{}

All records in a \textit{SA record set} must have the same 0x100 (secondary alignment) FLAG value.

\paragraph{}

All records in a \textit{SA record set} must have FLAG bit 0x4 (segment unmapped) not set.

\paragraph{}

All records in a \textit{SA record set} must align at least one read base that does not
overlap with any other alignments in the \textit{SA record set}.
That is, a chimeric alignment cannot contain superfluous alignment records.



- SEQ must match CIGAR
- Read lengths must be consistent


\end{document}

\title{SAM Strict Specification}
\author{Daniel L Cameron}
\date{\headdate}
\maketitle
\begin{quote}\small
The master version of this document can be found at
\url{https://github.com/samtools/hts-specs}.\\
This printing is version~\commitdesc\ from that repository,
last modified on the date shown above.
\end{quote}
\vspace*{1em}

\noindent
This document is a companion to the {\sl Sequence Alignment/Map Format
Specification} that defines the SAM file format.\footnote{See
\href{http://samtools.github.io/hts-specs/SAMv1.pdf}{\tt SAMv1.pdf} at \url{https://github.com/samtools/hts-specs}.}
The SAM file format defines the syntax required for a file to be
a valid SAM file. It does not require such files to be semantically
valid and internally consistent.
This document describes a set of additional semantic restrictions
for which the subset of syntactically valid SAM files that comply
with these restrictions can be described as \textit{SAM strict
compliant}.

\renewcommand{\abstractname}{Introduction}
\begin{abstract}

The SAM specifications have been instrumental in standardising
the file formats used for sequence alignment. A large ecosystem of
bioinformatics tools is now capable of reading and/or writing
SAM files. Unfortunately, many tools that read SAM files are tightly
coupled to a particular upstream tool 
and fail to correctly execute on valid SAM files written by other
tools. In part, this is due to the lack of semantic restrictions
inherent in the SAM file format. A syntactically valid SAM file
can be both internally inconsistent and semantically nonsensical.

The purpose of this document is to provide a baseline of semantic
validity for which tools should comply with when outputing SAM
files, and tools which input SAM files can safely assume when
they require input files to be \textit{SAM strict compliant}.

\end{abstract}

\section{Headers}
\samstrictrule{Undefined reserved header present}{Upper-case header record type codes are not defined in the SAM specifications must not be used.}
\samstrictrule{Undefined header tag present}{Upper-case header tags not defined in the SAM specifications must not be used.}
\samstrictrule{Tag present as both lowercase and uppercase}{A file should not contain the same tag in both upper-case and lowercase format. See the SAM specifications header tags best practice footnote.}
\samstrictrule{Malformed header}{Header lines with conform to either the {\tt
  /\char94@[A-Z][A-Z](\char92t[A-Za-z][A-Za-z0-9]:[
  -\char126]+)+\$/} or {\tt /\char94@CO\char92t.*/} or {\tt /\char94@CO\char92t.*/} regex.}{\samrule}

\subsection{HD}
\headerrequired{HD}
\headerunique{HD}
\headertagrequired{HD}{VN}{\samrule}
\samstrictrule{File does not start with HD header.}{The first header defined must be the HD header.}{\samrule}
\headertagregex{HD}{VN}{/\char94[0-9]+\char92.[0-9]+\$/}{\samrule}
\samstrictrule{Unknown SAM version}{The HD header VN tag version number must match a published version of the SAM specifications.}
\headertagvalues{HD}{SO}{{\tt unknown}, {\tt unsorted}, {\tt queryname} and {\tt coordinate}}{\samrule}
\headertagvalues{HD}{GO}{{\tt none}, {\tt query}, {\tt reference}}{\samrule}
\samstrictrule{Inconsistent HD header SO and GO tags}{The record orderings defined in the HD header SO and GO tags must be consistent}

\subsection{SQ}
\headertagrequired{SQ}{SN}{\samrule}
\headertagregex{SQ}{SN}{[!-)+-\char60\char62-\char126][!-\char126]*}{\samrule}
\headertagunique{SQ}{SN}
\headertagrequired{SQ}{LN}{\samrule}
\samstrictrule{Malformed SQ header LN tag}{The SQ header LN tag value must be an integer.}{\samrule}
\samstrictrule{Unsupported reference sequence length}{The SQ header LN tag value must greater than zero and less than 2147483648}{\samrule}
\headertagrequired{SQ}{M5}{\samrule}
\headertagmatchsamspecs{SQ}{AH}{\samrule}
\samstrictrule{Alternate locus references unknown reference sequence name}{Sequence names in the SQ header AH tag must match a SQ header SN reference sequence name.}{\samrule}
\headertagmatchsamspecs{SQ}{AN}{\samrule}
\samstrictrule{Duplicate alternative reference sequence names.}{Alternative reference sequence names defined in SQ header AN tags must be unique. A single tag cannot define cannot define the same alternative reference sequence name multiple times and multiple SQ headers cannot define the same alternative reference sequence name.}{\samrule}
\samstrictrule{Invalid alternative reference sequence names.}{Sequence names in the SQ header AN tag must not match any SQ header SN reference sequence names. }{\samrule}
\headertagregex{SQ}{M5}{[a-f0-9]\{32\}}

\subsection{RG}
\headertagrequired{RQ}{ID}{\samrule}
\headertagunique{RQ}{ID}{\samrule}
\samstrictrule{RQ header DT tag is not ISO8601}{RQ header DT tag must contain a valid date in ISO8601 format}{\samrule}
\headertagregex{RQ}{FO}{/\char92*|[ACMGRSVTWYHKDBN]+/}{\samrule}
\samstrictrule{Malformed RQ header PI tag}{The RQ header PI tag value must be a floating point value.}{\samrule}
\headertagvalues{RG}{PL}{{\tt CAPILLARY}, {\tt LS454}, {\tt ILLUMINA}, {\tt SOLID}, {\tt HELICOS}, {\tt IONTORRENT}, {\tt ONT}, and {\tt PACBIO}}{\samrule}
\samstrictrule{Invalid RG program group}{The RG header PG tag must contain one of the program groups specified in an ID tag of a PG header.}

\subsection{PG}
\headertagrequired{PG}{ID}{\samrule}
\headertagunique{PG}{ID}{\samrule}
\samstrictrule{Invalid PG header PP tag}{The PG header PP tag must contain one of the program groups specified in an ID tag of a PG header.}

\section{General}

\subsection{File Format}

\samstrictrule{File is not UTF-8}{The file must use UTF-8 encoding.}{\samrule}
\samstrictrule{Inconsistent line terminators}{All lines must be separated with the same new line character\(s\).}
\samstrictrule{Malformed floating point value}{All floating point values must conform to the regex {\tt [-+]?[0-9]*\char92.?[0-9]+([eE][-+]?[0-9]+)?}}
\samstrictrule{Malformed integer value}{All integer values must conform to the regex [-+]?[0-9]+}}

\subsection{Ordering}

\samstrictrule{Record ordering does not match HD header SO tag}{The order of records must be consistent with the HD header SO tag}
\samstrictrule{Record ordering does not match HD header GO tag}{The order of records must be consistent with the HD header GO tag}
\samstrictrule{Orphaned unmapped read}{If a read is unmapped, RNAME and POS must either be * and 0, or the RNAME and POS of another read from the same template.}

\section{Records}

\subsection{QNAME}
\samstrictrule{Empty QNAME}{QNAME fields must be at least one character in length}{\samrule}
\samstrictrule{QNAME too long}{QNAME fields must be at less than 255 characters in length}{\samrule}
\samstrictrule{Invalid character in QNAME}{QNAME fields must conform to the regex {TT \verb:[!-?A-~]}}{\samrule}

\subsection{FLAG}
\samstrictrule{Incorrect FLAG 0x1}{All templates with multiple segments must have FLAG 0x1 set}
\samstrictrule{Incorrect FLAG 0x1}{All templates with a single segment must not have FLAG 0x1 set}
\samstrictrule{Inconsistent FLAG 0x1}{All records with the same QNAME must have the same FLAG 0x1 value}
\samstrictrule{Inconsistent FLAG 0x2}{All primary records with the same QNAME must have the same FLAG 0x2 value}
\samstrictrule{Missing primary alignment record}{No supplementary or secondary alignments may exist for reads with an unmapped with primary alignment.}{\samrule}
\samstrictrule{Inconsistent FLAGs 0x1 0x2}{The 0x2 FLAG must not be set if the 0x1 FLAG is not.}
\samstrictrule{Inconsistent FLAGs 0x1 0x8}{The 0x8 FLAG must not be set if the 0x1 FLAG is not.}
\samstrictrule{Inconsistent FLAGs 0x1 0x20}{The 0x20 FLAG must not be set if the 0x1 FLAG is not.}
\samstrictrule{Inconsistent FLAGs 0x1 0x40}{The 0x40 FLAG must not be set if the 0x1 FLAG is not.}
\samstrictrule{Inconsistent FLAGs 0x1 0x80}{The 0x80 FLAG must not be set if the 0x1 FLAG is not.}
\samstrictrule{Inconsistent FLAGs 0x2 0x4}{The 0x2 FLAG must not be set if 0x4 is set in any primary alignments in the template}
\samstrictrule{Inconsistent FLAGs 0x4 0x8}{The 0x8 FLAG for primary record for each segment must match the 0x4 FLAG of the primary record for the next segment in the template}
\samstrictrule{Inconsistent FLAG 0x10 0x20}{The 0x20 FLAG must match the 0x10 FLAG for the primary alignment of the next segment in the template}
\samstrictrule{FLAG 0x20 set on unmapped read}{The 0x10 FLAG must not be set if the 0x4 FLAG is is set.}
\samstrictrule{Duplicate first segment primary records}{Of the primary records with the same QNAME, at most one can have FLAG 0x40 set}
\samstrictrule{Missing first segment primary record}{Of the records with the same QNAME with 0x1 FLAG, at least one record must have FLAG 0x40 set.}
\samstrictrule{Duplicate last segment primary records}{Of the primary records with the same QNAME, at most one can have FLAG 0x80 set}
\samstrictrule{Missing last segment primary record}{Of the records with the same QNAME with 0x1 FLAG, at least one record must have FLAG 0x80 set.}
\samstrictrule{Multiple primary alignment records}{Each segment must have at most one record with FLAG 0x100 and 0x800 not set.}{\samrule}
\samstrictrule{Missing primary alignment record}{Each segment must have at least one record with FLAG 0x100 and 0x800 not set.}{\samrule}
\samstrictrule{Unknown FLAG bit set}{FLAG bits higher than 0x800 must not be set}{\samrule}

\subsection{RNAME}
\samstrictrule{Malformed RNAME}{RNAME must conform to the regex {\tt \char92*|[!-()+-\char60\char62-\char126][!-\char126]*}}{\samrule}
\samstrictrule{RNAME not present in reference}{RNAME must be equal to the value of one of the SQ SN values defined in the header.}
\samstrictrule{RNAME contains character not supported by VCFv4.3}{RNAME must not contain any of the following characters: {\tt \char60\char62\char91\char93\char58\char42}}{\vcf43}
\samstrictrule{RNAME not supported by VCFv4.3}{RNAME must be not be one of {\tt DEL}, {\tt INS}, {\tt DUP}, {\tt INV}, {\tt CNV}, or {\tt BND}.}{\vcf43}

\subsection{POS}
\samstrictrule{Record placed outside of reference sequence}{If FLAG 0x4 is not set, POS must be between 0 and the length of the RNAME reference sequence inclusive. The length of the RNAME reference sequence can be found in the SQ header LN tag value for the SQ header with a SN tag matching the RNAME.}
\samstrictrule{Invalid POS}{POS cannot be 0 if FLAG 0x4 is set.}
\samstrictrule{Invalid POS}{POS cannot be negative.}{\samrule}
\samstrictrule{Invalid POS}{POS cannot be greater than 2147483647.}{\samrule}
\samstrictrule{POS specified without RNAME}{If RNAME is *, POS must be 0.}

\subsection{MAPQ}

\subsection{CIGAR}
\samstrictrule{Invalid CIGAR}{All CIGAR strings must conform to the regex {\tt \char92*|([0-9]+[MIDNSHPX=])+}}{\samrule}
\samstrictrule{Empty CIGAR}{All CIGAR strings must have at least one CIGAR operator}
\samstrictrule{Zero length CIGAR operator}{All CIGAR operators must have a non-zero positive length}
\samstrictrule{CIGAR contains operator repeat}{All adjacent CIGAR operators must be different.}
\samstrictrule{CIGAR does not contain any mapped bases}{All CIGARs must include a CIGAR operator that consumes a reference base.}{\samrule}
	{\tt Should we allow alignments with zero mapped bases? Seven bridges has a graph-based aligner that will
emit CIGARs such as 100I for alignments that align to a known insertion that is not included in the reference. Useful for local assembly but technical voilates the SAM specifications}
\samstrictrule{Incorrect CIGAR length}{Sum of lengths of the M/I/S/=/X operations must equal the length of SEQ when both CIGAR and SEQ are available.}{\samrule}
\samstrictrule{Invalid CIGAR hard clip}{H must only be present as the first and/or last operation.}{\samrule}
\samstrictrule{Invalid CIGAR soft clip}{S must only have H operations between them and the ends of the CIGAR string.}{\samrule}
\samstrictrule{CIGAR overhangs reference sequence}{POS plus the number of reference bases consumed by the CIGAR must not exceed the length of the RNAME reference sequence.}
\samstrictrule{Inconsistent CIGAR read lengths}{All mapped alignments for a given segment must have matching read lengths. That is, the sum of lengths of the M/I/S/=/X/H operations must be equal.}

\samstrictrule{Unusual indel positioning}{TODO: should we disallow I/D operators at the ends of reads? There was some ambiguity as to how deletions interacted with POS but I think the spec has been updated in favour of the BWA interpretation since that discussion.}

\subsection{RNEXT}
\samstrictrule{Invalid RNEXT}{If the template contains one segment RNEXT must be *}
\samstrictrule{RNEXT not present in reference}{RNEXT must be equal to the value of one of the SQ SN values defined in the header.}
\samstrictrule{Invalid RNEXT}{If the primary alignment of the next read in the template is mapped, RNEXT must not be {\tt *}}{\samrule}
\samstrictrule{RNEXT not using =}{If the primary alignment of the next read in the template is mapped to the same reference sequence, RNEXT must be {\tt =}}{\samrule}
\samstrictrule{Incorrect RNEXT}{If this read is unmapped or the primary alignment of the next read in the template is mapped to the a different reference sequence, RNEXT must match the RNAME of the next read.}

\subsection{PNEXT}
\samstrictrule{Invalid PNEXT}{If the template contains one segment PNEXT must be 0}
\samstrictrule{Incorrect PNEXT}{If the primary alignment of the next read in the template is unmapped, PNEXT must be 0}
\samstrictrule{Incorrect PNEXT}{If the primary alignment of the next read in the template is mapped, PNEXT must match the POS of that record.}

\subsection{TLEN}
TODO: can we actually do anything for this?

\subsection{SEQ}
\samstrictrule{Inconsistent SEQ read lengths}{All alignments of a given segment must have consistent SEQ lengths. That is, for all non-* SEQ, SEQ + length of CIGAR hard clip must be equal. }
\samstrictrule{Inconsistent SEQ sequences}{All alignments of a given segment must have consistent base calls. A base cannot be called an A in one record, but a T in another. Note that to determine the read base, both the 0x10 FLAG, and any hard clipping CIGAR operators need to be taken into account.}

TODO: I am up to here

\subsection{QUAL}















































\paragraph{}


\paragraph{}




\subsection{TODO: categorize}

- secondary alignment mate info must match primary
- supplementary alignment mate info must match primary

\paragraph{}

All read alignments must have CIGARs with matching read length. That is,
the sum of lengths of the M/I/S/=/X operations must be equal for all mapped read alignment.
This means that chimeric alignments must include either soft clipping or
hard clipping CIGAR operations for read bases which were not aligned.

\paragraph{}

All read alignments with SEQ not equal to * must have a SEQ consistent
with all other read alignments. That is, chimeric and secondary alignments that have non-* SEQ
must be consistent with all other records defining a segment sequence. For
example, 10th base in a read cannot be be A in one alignment, but T in another.

\paragraph{}

All read alignments with QUAL not equal to * must have a SEQ consistent
with all other read alignments.

\subsection{Mate alignments}

\paragraph{}

Unmapped reads must have RNAME and POS identical to that of the primary
non-supplementary alignment of first mapped read from the originating template.
For paired-end or mate-pair sequencing, this equivalent
to setting the RNAME/POS of unmapped reads to the RNAME/POS of the mate.

\paragraph{}

For templates with multiple reads, RNEXT and PNEXT must match the
alignment of the non-supplementary primary alignment of the next read.
As per the SAM specifications, for the last read, the next read is the first read in the template.

\paragraph{}

If all segments in a template are unmapped, RNAME must be set as `*' and POS as 0.

\subsection{Reference bounds}

\paragraph{}

Read alignments must not extend past the start or end of the aligned RNAME.

\subsection{Mapping Qualities}

\paragraph{}

All mapping quality scores, including those defined in tags must be within the range [0, 255].
A value 255 indicates that the mapping quality is not available and must only be used if the
mapping quality field is required. For example, a mapping quality field value is required for
MAPQ field and the mapq portion of the SA tag, but as the AM is optional, a mapping quality
field value is not required and the AM tag should be omitted entirely if a mapping quality is
not available.

\section{SAM Tags}

\subsection{Standard Tags}

\paragraph{}

No record can include any reserved tags not defined in the
{\sl Sequence Alignment/Map Optional Fields Specification}.
Non-standard tags must start X, Y, Z or a lowercase letter as per the SAM specifications.

\paragraph{}

The \textit{type} of all \textit{standard tags} must match the type
defined in the {\sl Sequence Alignment/Map Optional Fields Specification}.

\paragraph{}

All tag values must be consistent with the format
defined in the {\sl Sequence Alignment/Map Optional Fields Specification}.

\subsection{SA}

For the purpose of this section, a \textit{SA record set} is a set of SAM records
from a single \textit{read} which collectively represent a single \textit{chimeric alignment}.

\samstrictrule{Missing SA tag}{All records with FLAG 0x800 set must have a SA tag defined.}{\samrule}
\samstrictrule{Missing non-supplementary chimeric alignment record}{Each chimeric alignment must have a record with FLAG 0x800 not set.}

\paragraph{}

All SA tag values must satisfy the SA tag regular expression
defined in the {\sl Sequence Alignment/Map Optional Fields Specification}.

\paragraph{}

All records referenced in the SA tag of a given record must have a SA tag defined.

\paragraph{}

All records referenced in the SA tag of a given record must include the given
record in their SA tag.

\paragraph{}

All records in a \textit{SA record set} with a FI tag defined must have the same FI tag value.

TODO: which other tags?

\paragraph{}

All records referenced in the SA tag must exist with matching \textit{rname, pos, strand, CIGAR}.

\paragraph{}

The SA \textit{mapq} of all references to a given record in a \textit{SA record set} must
match the record mapping quality.

\paragraph{}

The SA \textit{NM} of all references to a given record in a \textit{SA record set} must
match the record \textit{NM} tag value.

\paragraph{}

All records except 1 in a \textit{SA record set} must have the 0x800 (supplementary alignment) FLAG bit set.

\paragraph{}

The first SA record in all supplementary alignment records must be the canonical non-supplementary alignment.

\paragraph{}

All records in a \textit{SA record set} must have the same 0x100 (secondary alignment) FLAG value.

\paragraph{}

All records in a \textit{SA record set} must have FLAG bit 0x4 (segment unmapped) not set.

\paragraph{}

All records in a \textit{SA record set} must align at least one read base that does not
overlap with any other alignments in the \textit{SA record set}.
That is, a chimeric alignment cannot contain superfluous alignment records.



- SEQ must match CIGAR
- Read lengths must be consistent


\end{document}

\title{Sequence Alignment/Map Strict Specification}
\author{The SAM/BAM Format Specification Working Group}
\date{\headdate}
\maketitle
\begin{quote}\small
The master version of this document can be found at
\url{https://github.com/samtools/hts-specs}.\\
This printing is version~\commitdesc\ from that repository,
last modified on the date shown above.
\end{quote}
\vspace*{1em}

\noindent
This document is a companion to the {\sl Sequence Alignment/Map Format
Specification} that defines the SAM file format.\footnote{See
\href{http://samtools.github.io/hts-specs/SAMv1.pdf}{\tt SAMv1.pdf} at \url{https://github.com/samtools/hts-specs}.}
The SAM file format defines the syntax required for a file to be
a valid SAM file. It does not require such files to be semantically
valid and internally consistent.
This document describes a set of additional semantic restrictions
for which the subset of syntactically valid SAM files that comply
with these restrictions can be described as \textit{SAM strict
compliant}.

\renewcommand{\abstractname}{Introduction}
\begin{abstract}

The SAM specifications have been instrumental in standardising
the file formats used for sequence alignment. A large ecosystem of
bioinformatics tools is now capable of reading and/or writing
SAM files. Unfortunately, many tools that read SAM files are tightly
coupled to a particular upstream tool 
and fail to correctly execute on valid SAM files written by other
tools. In part, this is due to the lack of semantic restrictions
inherent in the SAM file format. A syntactically valid SAM file
can be both internally inconsistent and semantically nonsensical.

The purpose of this document is to provide a baseline of semantic
validity for which tools should comply with when outputing SAM
files, and tools which input SAM files can safely assume when
they require input files to be \textit{SAM strict compliant}.

\end{abstract}

\section{Headers}

\section{General}

\subsection{CIGAR}

\paragraph{}

All CIGAR operators must have a non-zero positive length

\paragraph{}

All adjacent CIGAR operators must be different.

\subsection{Mate alignments}

\paragraph{}

Unmapped reads must have the RNAME and POS identical to that of the first mapped read
from the originating template. For paired-end or mate-pair sequencing, this equivalent
to setting the RNAME/POS of unmapped reads to the RNAME/POS of the mate.

\paragraph{}

If all segments in a template are unmapped, RNAME must be set as `*' and POS as 0.

\subsection{Reference bounds}

\paragraph{}

Read alignments must not extend past the start or end of the aligned RNAME.

\section{SAM Tags}

\subsection{Standard Tags}

\paragraph{}

No record can include any reserved tags not defined in the
{\sl Sequence Alignment/Map Optional Fields Specification}.
Non-standard tags must start X, Y, Z or a lowercase letter as per the SAM specifications.

\paragraph{}

The \textit{type} of all \textit{standard tags} must match the type
defined in the {\sl Sequence Alignment/Map Optional Fields Specification}.

\subsection{SA}

For the purpose of this section, a \textit{SA record set} is a set of SAM records
from a single \textit{read} which collectively represent a single \textit{chimeric alignment}.

\paragraph{}

All records referenced in the SA tag of a given record must have a SA tag defined.

\paragraph{}

All records referenced in the SA tag of a given record must include the given
record in their SA tag.

\paragraph{}

All records in a \textit{SA record set} with a FI tag defined must have the same FI tag value.

\paragraph{}

All records referenced in the SA tag must exist with matching \textit{rname, pos, strand, CIGAR}.

\paragraph{}

The SA \textit{mapq} of all references to a given record in a \textit{SA record set} must
match the record mapping quality.

\paragraph{}

The SA \textit{NM} of all references to a given record in a \textit{SA record set} must
match the record \textit{NM} tag value.

\paragraph{}

All records except 1 in a \textit{SA record set} must have the 0x800 (supplementary alignment) FLAG bit set.

\paragraph{}

The first SA record in all supplementary alignment records must be the canonical non-supplementary alignment.

\paragraph{}

All records in a \textit{SA record set} must have the same 0x100 (secondary alignment) FLAG bit.

\paragraph{}

All records in a \textit{SA record set} must have FLAG bit 0x4 (segment unmapped) not set.

\paragraph{}

All records in a \textit{SA record set} must have the same RNEXT.

\paragraph{}

All records in a \textit{SA record set} must have the same PNEXT.

\paragraph{}

All records in a \textit{SA record set} must have CIGARs with matching read length. That is,
chimeric alignment records must include either soft clipping or hard clipping CIGAR operations
for read bases which were not aligned.

\paragraph{}

All records in a \textit{SA record set} with SEQ not equal to * must have a SEQ consistent
with all other records in the \textit{SA record set}. That is, chimeric alignments that define
a segment sequence must be consistent with all other records defining a segment sequence. For
example, 10th base in a read cannot be be A in one chimeric alignment, but T in another.

\paragraph{}

All records in a \textit{SA record set} with QUAL not equal to * must have a SEQ consistent
with all other records in the \textit{SA record set}.

\paragraph{}

All records in a \textit{SA record set} must align at least one read base that does not
overlap with any other alignments in the \textit{SA record set}.
That is, a chimeric alignment cannot contain superfluous alignment records.

\end{document}
